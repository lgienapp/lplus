% --------------------------------------------------------------------
% Name           : slides.tex
% Author         : Lukas Gienapp
% Version        : 0.1
% License        : This file may be distributed and/or modified under 
%				           the MIT License.
% Description    : A simplistic, feature-complete Beamer theme.
%---------------------------------------------------------------------
\documentclass[aspectratio=169]{beamer}
\usetheme{lplus}

\usepackage{booktabs}
\usepackage{tikz}

\tikzset{
    invisible/.style={opacity=0,text opacity=0},
    visible on/.style={alt=#1{}{invisible}},
    alt/.code args={<#1>#2#3}{%
      \alt<#1>{\pgfkeysalso{#2}}{\pgfkeysalso{#3}} 
    }
}


\newcommand{\themename}{\textbf{lplus}~}

\title[L+]{lplus}
\subtitle[Beamer Theme]{A minimalist Beamer theme}
\author[Author One et al.]{Author One \inst{1} \and Author Two \inst{1} \and Author Three \inst{2}}
% Affiliations
\institute{\inst{1} First institute \and \inst{2} Second institute}
\date{\today}
\titlegraphic{../assets/lplus.pdf}

\begin{document}

\maketitle

%\begin{frame}{Table of contents}
%  \setbeamertemplate{section in toc}[sections numbered]
%  \tableofcontents
%\end{frame}

\section{Introduction}

\begin{frame}[fragile]{lplus}
\themename  is a simplistic, feature-complete Beamer theme.

Enable the theme by loading
\begin{verbatim}
    \documentclass{beamer}
    \usetheme{lplus}
\end{verbatim}
\end{frame}

\section{Options}

\begin{frame}[fragile]{Package Options}
Besides the standard beamer package options (such as \alert{\texttt{aspect}}), \themename introduces several custom package options:
\begin{itemize}
\item \alert{\texttt{darkmode}} uses an inverted color scheme
\item \alert{\texttt{nosectionpages}} and \alert{\texttt{nosubsectionpages}} hides section or only subsection pages. If section pages are hidden, no subsections are shown regardless of whether \alert{\texttt{nosubsectionpages}} is set.
\item \alert{\texttt{truewhite}} reverts to pure white and black; the theme uses a slight off-white and gray by default, which can cause problems when, e.g., printing
\item \alert{\texttt{colorblocks}} renders block environments with background colors. By default, only the outline of a block is drawn.
\end{itemize}
\end{frame}

\begin{frame}[fragile]{Title Page Macros}
The usual title page macros such as author, date, title and subtitle are supported.
Additional options include:
\begin{itemize}
\item Short title for use in the header (analogous for subtitle) \\
\verb+\title[short title]{long title}+ 
\item Institution \\ 
\verb+\institute{institution}+
\item Additional graphics on the title page: \\ 
\verb+\titlegraphic{filename}+
\end{itemize}
\end{frame}

\section{Typography}

\begin{frame}[fragile]{Typography}
\themename uses the marvellous \href{https://rsms.me/inter/}{\texttt{inter}} font family, paired with \href{https://levien.com/type/myfonts/inconsolata.html}{\texttt{inconsolata}} for monospaced text.
\begin{verbatim}
You can \emph{emphasize} text, \alert{accent} 
parts or show \textbf{bold} results.
\end{verbatim}
\begin{center}becomes\end{center}
You can \emph{emphasize} text, \alert{accent} parts or show \textbf{bold} results.\footnote{Footnotes work as well. Even in \texttt{monospace}.}
\end{frame}

\begin{frame}{Font features}
  \begin{itemize}
    \item Regular
    \item \textit{Italic}
    \item \textbf{Bold}
    \item \textbf{\textit{Bold Italic}}
    \item \texttt{Monospace}
    \item \texttt{\textit{Monospace Italic}}
    \item \texttt{\textbf{Monospace Bold}}
    \item \texttt{\textbf{\textit{Monospace Bold Italic}}}
  \end{itemize}
\end{frame}

\begin{frame}{Lists}
  \begin{columns}[T,onlytextwidth]
    \column{0.33\textwidth}
      Items
      \begin{itemize}
        \item Item \item Item \item Item
      \end{itemize}

    \column{0.33\textwidth}
      Enumerations
      \begin{enumerate}
        \item First \item Second  \item Last
      \end{enumerate}

    \column{0.33\textwidth}
      Descriptions
      \begin{description}
        \item[Caption] Text 
        \item[Caption] Text
        \item[Caption] Text
      \end{description}
  \end{columns}
\end{frame}

\begin{frame}{Math}
  \begin{equation*}
    \frac{1}{\sigma\sqrt{2\pi}}\exp\biggl(\frac{-x^2}{2\sigma^2}\biggr)
  \end{equation*}
\end{frame}

\begin{frame}[fragile]{Citations}
For a general introduction to creating slides with Beamer, refer to the Beamer manual.\cite{BeamerManual}
\end{frame}


\section{Environments}
\begin{frame}{Environments}
\themename defines the standard beamer environments:
\begin{itemize}
    \item Figures
    \item Tables
    \item Blocks
\end{itemize}
\end{frame}

\subsection{Figures}
\begin{frame}{Figures}
  \begin{figure}
    \begin{tikzpicture}[scale=1.25]
   	\draw [help lines] (-4.25,-1.25) grid (4.25,1.5);
   	\draw [help lines,step=0.25cm] (-2.99,0) grid (2.99,0.99);
   \begin{scope}[smooth,draw=gray!20,y=0.3989422804cm]
        \filldraw [fill=Blue5] plot[id=f1,domain=-3:-2] function {exp(-x*x/2)}
            -- (-2,0) -- (-3,0) -- cycle;
        \filldraw [fill=Blue3] plot[id=f2,domain=-2:-1] function {exp(-x*x/2)}
            -- (-1,0) -- (-2,0) -- cycle;
        \filldraw [fill=Blue1] plot[id=f3,domain=-1:0]  function {exp(-x*x/2)}
            -- (0,0)  -- (-1,0) -- cycle;
        \filldraw [fill=Blue1] plot[id=f4,domain=0:1] function {exp(-x*x/2)}
            -- (1,0)  --  (0,0) -- cycle;
        \filldraw [fill=Blue3] plot[id=f5,domain=1:2] function {exp(-x*x/2)}
            -- (2,0)  -- (1,0) -- cycle;
        \filldraw [fill=Blue5] plot[id=f6,domain=2:3] function {exp(-x*x/2)}
            -- (3,0)  -- (2,0) -- cycle;
        \draw[black] plot[id=f7,domain=-4.25:4.25,samples=100]
            function {exp(-x*x/2)};
   \end{scope}
       \draw[->] (-4.25,0) -- (4.25,0) node [right] {$x$};

    \foreach \pos/\label in {-3/$-3\sigma$,-2/$-2\sigma$,-1/$-\sigma$,
            1/$\sigma$,2/$2\sigma$,3/$3\sigma$}
        \draw (\pos,0) -- (\pos,-0.1) (\pos cm,-3ex) node
            [anchor=base,inner sep=1pt]  {\label};

    \draw (-0.1,1) -- (.1,1) node [right,inner sep=1pt] {$\sigma$};

    \foreach \pos/\percent/\height in {1/34/0.5,2/14/0.25,3/2/0.125,4/0.1/0.1}
    {
      \node[text=Blue\pos,anchor=base,yshift=2pt,xshift=-0.625cm,
        inner sep=1pt] at (\pos,\height) {$\percent\%$};
      \node[text=Blue\pos,anchor=base,yshift=2pt,xshift=.625cm,
        inner sep=1pt]  at (-\pos,\height) {$\percent\%$};
    }
\end{tikzpicture}
    \caption{Standard deviation from \href{http://texample.net/tikz/examples/standard-deviation/}{texample.net}.}
  \end{figure}
\end{frame}

\subsection{Tables}
\begin{frame}{Tables}
  \begin{table}
    \caption{Largest cities in the world (source: Wikipedia)}
    \begin{tabular}{@{} lr @{}}
      \toprule
      City & Population\\
      \midrule
      Mexico City & \texttt{20,116,842}\\
      Shanghai    & \texttt{19,210,000}\\
      Peking      & \texttt{15,796,450}\\
      Istanbul    & \texttt{14,160,467}\\
      \bottomrule
    \end{tabular}
  \end{table}
\end{frame}

\subsection{Blocks}
\begin{frame}{Blocks}
  	Five different block environments are pre-defined. The \texttt{colorblocks} option can be invoked to render solid block background colors.
   
    \begin{block}{Default}
    Block content.
    \end{block}

    \begin{alertblock}{Alert}
    Block content.
    \end{alertblock}
    
    \begin{exampleblock}{Example}
    Block content.
    \end{exampleblock}

    \begin{remark}{Remark}
    Block content.
    \end{remark}
    
    \begin{standout}[4cm]
    Standout content with specified width.
    \end{standout}
\end{frame}

\section{Colors}
\begin{frame}{Colors}

\themename defines a color-blind friendly theme of 5 colors in 6 perceptually uniform variants each. Colors are automatically inverted in dark mode. The white and black used by the theme are defined as \texttt{Black} and \texttt{White}.

\begin{center}
    \begin{tikzpicture}
        % Draw Examples
        \foreach \color/\y  in {
            Gray/-1, % Gray
            Blue/-2, % Blue
            Red/-3, % Red
            Purple/ -4, % Purple
            Green/ -5, % Green
            Yellow/-6  % Yellow    
        }{%
            \foreach \index/\x in {
                1/0,
                2/2,
                3/4,
                4/6,
                5/8,
                6/10
             }{
                \node[
                    rectangle, 
                    minimum width=10ex,
                    minimum height=2ex,
                    fill=\color\index,
                    label={\scriptsize\texttt{\color\index}}
                ] at (\x, \y) {};
            }
           
        }%
    \end{tikzpicture}
\end{center}
\end{frame}

\begin{frame}{Color Gradients}

\begin{columns}
    \begin{column}{.49\textwidth}
    The six variants in each color are equally spaced in brightness. They can be linearly interpolated to approximate perceptually uniform gradients. 
    \vspace{1em}
    
    The use of colors beyond the \texttt{1} and \texttt{6} variants is discouraged as lighter and darker colors are too hard to tell apart.
    \end{column}

    \begin{column}{.49\textwidth}
        \usepgflibrary {shadings}
        \begin{center}
        \begin{tikzpicture}
        
        \foreach \i/\color in {1/Red, 2/Purple, 3/Blue, 4/Green, 5/Yellow, 6/Gray}{
            %\draw[left color=Black, right color=\color1, draw=none] (-1.3,\i) rectangle (1,\i+1); 
            \foreach \j/\k in {1/2,2/3,3/4,4/5,5/6}{
                 \draw[left color=\color\j, right color=\color\k, draw=none] (\j,\i) rectangle (\k,\i+1);    
             }
             %\draw[left color=\color6, right color=White, draw=none] (6,\i) rectangle (6.54,\i+1);    
        }
        \foreach \i in {1,2,3,4,5,6}{
             \path [draw, line width=.25pt, color=White] (\i, 0) -- (\i, 7);
             \node[text=Black] at (\i, .75) {\i};
        }
        %\draw[fill=none, draw=Black] (-1.3, 1) rectangle (6.54, 7);
        \end{tikzpicture}
        \end{center}
    \end{column}
\end{columns}
\end{frame}


\section*{References}
\begin{frame}[allowframebreaks]
  \bibliographystyle{plain}
	\begin{thebibliography}
    @misc{BeamerManual,
      title = {{The beamer class} - User Guide for version 3.50.},
      howpublished = {\url{http://tug.ctan.org/macros/latex/contrib/beamer/doc/beameruserguide.pdf}},
      note = {Accessed: 2018-04-05}
    }
  \end{thebibliography}
\end{frame}


\end{document}
